% Turing operation: write, move, next state
\newcommand{\Top}[3]{%
	#1\ifstrequal{#2}{R}{%
	\tikz[baseline=0]{\draw[->] (0,.66ex) -- ++(2ex,0);}}{%
	\tikz[baseline=0]{\draw[<-] (0,.66ex) -- ++(2ex,0);}}%
	#3}%
\hfill\subbottom[Add algorithm]{%
	\begin{minipage}[b]{.4\linewidth}
	\hfil\begin{tabular}[b]{ccc}
	\toprule
			& \multicolumn{2}{c}{tape input}		\\
	state	& 0					& 1					\\ \midrule
	A		& \Top{1}{R}{B}		& \Top{1}{R}{A}		\\
	B		& \Top{0}{L}{C}		& \Top{1}{R}{B}		\\
	C		& \Top{0}{R}{HALT}	& \Top{0}{R}{HALT}	\\ \bottomrule
	\end{tabular}\hfil \\
	\footnotesize\figureversion{lining,prop}\Top{1}{R}{B} means: write 1, move head to right, go to state B%
	\end{minipage}%
}\hfill\hfill%
\subbottom[Execution steps to add 2 and 3]{%
	\begin{tikzpicture}[
		bit/.style={draw,minimum size=\portablefigunit,inner sep=0,outer sep=0,fontB,execute at end node={\gdef\Tbitstyle{}}},
		head/.style={bit,fill=black!15,alias=last head,execute at end node={\gdef\Tbitstyle{\itshape}}},
		first bit/.style={bit,draw=none,minimum width=0,name=last bit,alias=last sequence,outer ysep=.2ex},
		bit A/.style={head},
		bit B/.style={head},
		bit C/.style={head},
		bit H/.style={head},
		bit ./.style={draw=none},
	]
	\def\strut{}%\rule[0pt]{0pt}{.8\portablefigunit}}
	\def\Tgobble#1{}
	\def\Tbitnode#1{\node[bit,bit #1/.try,anchor=west,at=(last bit.east)] (last bit) {\strut\ifstrequal{#1}{.}{}{\Tbitstyle#1}};\Tbit}
	\def\Tbit#1{\ifx\relax#1\let\Tbitnode\Tgobble\fi\Tbitnode#1}
	\def\Tstep#1;{
		\node[first bit,anchor=north,at=(last sequence.south)] {\strut};
		{\Tbit#1\relax}
	}
	\gdef\Tbitstyle{}
	\node[first bit,anchor=south] at (0,0) {\strut};
%
	\Tstep ......A1101110;
	\draw[<-,shorten >=1pt] (last head.north) -- +(.5,1em) node[anchor=south,plain label,outer sep=-1pt] (head label) {head: state and next input};
	\draw[->,shorten <=1pt] (head label.south) -- +(.5,-1em);
	\Tstep .....1A101110;
	\Tstep ....11A01110;
	\Tstep ...111B1110;
	\Tstep ..1111B110;
	\Tstep .11111B10;
	\Tstep 111111B0;
	\Tstep .11111C10;
	\Tstep 111110H0;
%
	\coordinate (bb north west) at (current bounding box.north west);
	\coordinate (bb south west) at (current bounding box.south west);
	\coordinate (bb south east) at (current bounding box.south east);
	\draw[->,semithick] let \p1=($(bb north west)+(-1ex,0)$), \p2=($(bb south west)+(-1ex,0)$) in
		(\x1,0) -- (\p2) node[plain label,midway,anchor=south,outer sep=.5ex,rotate=90] {execution};
	\draw[<->,semithick] let \p1=($(bb south west)+(0,-1ex)$), \p2=($(bb south east)+(0,-1ex)$) in
		(\p1) -- (\p2) node[plain label,midway,anchor=north,outer sep=.5ex] {tape};
	\end{tikzpicture}%
}\hfill\null%
